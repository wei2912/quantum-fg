\chapter{Postulates of Quantum Mechanics}

\section{State space}

\begin{postulate}
  Associated to any isolated physical system is a complex vector space with
  inner product (that is, a Hilbert space) known as the \emph{state space} of
  the system. The system is completely described by its \emph{state vector},
  which is a unit vector in the system's state space.
\end{postulate}

The simplest quantum mechanical system, and the system which we will be most
concerned with, is the \emph{qubit}. A qubit has a two-dimensional state space.
Suppose $\ket{0}$ and $\ket{1}$ form an orthonormal basis for that state space.
Then an arbitrary state vector in the state space can be written
\begin{equation*}
  \ket{\psi} = a\ket{0} + b\ket{1},
\end{equation*} where $a$ and $b$ are complex numbers. The condition that
$\psi$ be a unit vector, $\ketbra{\psi}{\psi} = 1$, is therefore equivalent to
$|a|^2 + |b|^2 = 1$. The condition $\ketbra{\psi}{\psi} = 1$ is often known as
the \emph{normalization condition} for state vectors.

\paragraph{Superposition.} We say that any linear combination $\sum_i \alpha_i
\ket{\psi_i}$ is a \emph{superposition} of the states $\ket{\psi_i}$ with
\emph{amplitude} $\alpha_i$ for the state $\ket{\psi_i}$.

\section{Evolution}

\begin{postulate}
  The evolution of a \emph{closed} quantum system is described by a
  \emph{unitary transformation}. That is, the state $\ket{\psi}$ of the system
  at time $t_1$ is related to the state $\ket{\psi'}$ of the system at time
  $t_2$ by a unitary operator $U$ which depends only on the times $t_1$ and
  $t_2$, \begin{equation*}
    \ket{\psi'} = U\ket{\psi}.
  \end{equation*}

  The evolution of the state of a \emph{closed} quantum system is also
  described by the \emph{Schr{\"o}dinger equation}, \begin{equation}
    i\hbar\frac{\mathrm{d}\ket{\psi}}{\mathrm{d}t} = H\ket{\psi}.
      \label{eq:sch-eq}
  \end{equation}
\end{postulate}

In Schr{\"o}dinger's equation, $\hbar$ is a physical constant known
as \emph{Planck's constant} whose value must be experimentally determined. The
exact value is not important to us. In practice, it is common to absorb the
factor $\hbar$ into $H$, effectively setting $\hbar = 1$. $H$ is a fixed
Hermitian operator known as the \emph{Hamiltonian} of the closed system.

Because the Hamiltonian is a Hermitian operator it has a spectral decomposition
\begin{equation}
  H = \sum_E E\ketbra{E}{E}, \label{eq:ham-spec-dec}
\end{equation} with eigenvalues $E$ and corresponding normalized eigenvectors
$\ket{E}$. The states $\ket{E}$ are conventionally referred to as \emph{energy
eigenstates}, or sometimes as \emph{stationary states}, and $E$ is the
\emph{energy} of the state $\ket{E}$. The lowest energy is known as the
\emph{ground state energy} for the system, and the corresponding energy
eigenstate (or eigenspace) is known as the \emph{ground state}. The reason the
states $\ket{E}$ are sometimes known as stationary states is because their only
change in time is to acquire an overall numerical factor, \begin{equation*}
  \ket{E} \rightarrow \exp(-iEt/\hbar)\ket{E}.
\end{equation*}
The solution to Sch{\"o}dinger's equation will be verified later to be:
\begin{equation*} \ket{\psi(t_2)} = \exp\left[\frac{-iH(t_2 - t_1)}{\hbar}\right]
    \ket{\psi(t_1)} = U(t_1, t_2)\ket{\psi(t_1)},
\end{equation*} where we define \begin{equation}
  U(t_1, t_2) \equiv \exp\left[\frac{-iH(t_2 - t_1)}{\hbar}\right].
    \label{eq:unit-sch-eq}
\end{equation}

\paragraph{\cite{mikeandike} Exercise 2.54:} Suppose $A$ and $B$ are commuting
Hermitian operators. Prove that $\exp(A)\exp(B) = \exp(A + B)$.

\paragraph{Solution:} Since $A$ and $B$ are commuting Hermitian operators, they
are simultaneously diagonalizable. We write $A = \sum_i a_i\ketbra{i}{i}$,
$B = \sum_j b_j\ketbra{j}{j}$, noting that $A + B = \sum_i (a_i + b_i)
\ketbra{i}{i}$. Then, $\exp(A) = \sum_i e^{a_i}\ketbra{i}{i}$, $\exp(B) =
\sum_j e^{b_j}\ketbra{j}{j}$, $\exp(A + B) = \sum_i e^{a_i + b_i}
\ketbra{i}{i}$. It is clear that \begin{align*}
  \exp(A)\exp(B)
    &= \left(\sum_i e^{a_i}\ketbra{i}{i}\right)\left(\sum_j e^{b_j}
      \ketbra{j}{j}\right) \\
    &= \sum_{i, j} e^{a_i}e^{b_j}\ket{i}\delta_{ij}\bra{j} \\
    &= \sum_i e^{a_i}e^{b_i}\ketbra{i}{i} \\
    &= \sum_i e^{a_i + b_i}\ketbra{i}{i} = \exp(A + B),
\end{align*} which proves the result.

\paragraph{\cite{mikeandike} (modified) Exercise 2.55:} Prove that $U(t_1,
t_2)$ as defined in \eqref{eq:unit-sch-eq} is unitary.

\paragraph{Solution:} Consider $U(t_1, t_2) = \exp\left[\frac{-iH(t_2 - t_1)}
{\hbar}\right]$ and $U(t_1, t_2)^\dagger = \exp\left[\frac{iH^\dagger(t_2 -
t_1)}{\hbar}\right]$. Rewriting $U(t_1, t_2) = \exp(A)$ and $U(t_1,
t_2)^\dagger = \exp(B)$ where $A$ and $B$ are Hermitian operators, it is clear
that $A = -B$ and commute. Hence, $\exp(A)\exp(B) = \exp(A + B) = I$.

\paragraph{\cite{mikeandike} Exercise 2.56:} Use the spectral decomposition to
show that $K \equiv -i\log(U)$ is Hermitian for any unitary $U$, and thus $U =
\exp(iK)$ for some Hermitian $K$.

\paragraph{Solution:} Noting that all eigenvalues of $U$ are of modulus 1, one
could express $U = \sum_i e^{i\theta_i}\ketbra{i}{i}$ with the spectral
decomposition theorem, leading to $K = -i\log(U) = \sum_i \theta_i\ketbra{i}
{i}$. Clearly, $K = K^\dagger$, so $K$ is Hermitian.

\paragraph{}

These exercises show that $U(t_1, t_2)$ is unitary. There is therefore a
one-to-one correspondence between the discrete-time description of dynamics
using unitary operators, and the continuous time description using
Hamiltonians.

\section{Quantum measurement}

\begin{postulate}
  Quantum measurements are described by a collection $\{M_m\}$ of
  \emph{measurement operators}. These are operators acting on the state space
  of the system being measured. The index $m$ refers to the measurement
  outcomes that may occur in the experiment. If the state of the quantum
  system is $\ket{\psi}$ immediately before the measurement then the
  probability that result $m$ occurs is given by \begin{equation*}
    p(m) = \mel{\psi}{M_m^\dagger M_m}{\psi},
  \end{equation*} and the state of the system after the measurement is
  \begin{equation*}
    \frac{M_m\ket{\psi}}{\sqrt{\mel{\psi}{M_m^\dagger M_m}{\psi}}}.
  \end{equation*}
  The measurement operators satisfy the \emph{completeness equation},
  \begin{equation*}
    \sum_m M_m^\dagger M_m = I.
  \end{equation*}
  The completeness equation expresses the fact that probabilities sum to one:
  \begin{equation*}
    1 = \sum_mp(m) = \sum_m\mel{\psi}{M_m^\dagger M_m}{\psi}.
  \end{equation*}
\end{postulate}

\paragraph{\cite{mikeandike} Exercise 2.57: (Cascaded measurements are single
measurements)} Suppose $\{L_l\}$ and $\{M_m\}$ are two sets of measurement
operators. Show that a measurement defined by the measuerment operators
$\{L_l\}$ followed by a measurement defined by the measurement operators
$\{M_m\}$ is physically equivalent to a single measurement defined by
measurement operators $\{N_{lm}\}$ with the representation $N_{lm} \equiv M_m
L_l$.

\paragraph{Solution:} Take $\ket{\psi}$ to be the state of the quantum system
immediately before both measurements. Then, the state of the system after
measurement by $L_l$ is \begin{equation*}
  \ket{\psi'} = \frac{L_l\ket{\psi}}{\sqrt{\mel{\psi}{L_l^\dagger L_l}{\psi}}},
\end{equation*} and the state of the system after measurement by $M_m$ is
\begin{equation*}
  \ket{\psi''} = \frac{M_m\ket{\psi'}}{\sqrt{\mel{\psi'}{M_m^\dagger M_m}
    {\psi'}}}.
\end{equation*}
Noting that $\mel{\psi'}{M_m^\dagger M_m}{\psi'} = \frac{\mel{\psi}{L_l^\dagger
M_m^\dagger M_m L_l}{\psi}}{\mel{\psi}{L_l^\dagger L_l}{\psi}}$,
\begin{align*}
  \ket{\psi''} &= \left(M_m\frac{L_l\ket{\psi}}{\sqrt{\mel{\psi}{L_l^\dagger
    L_l}{\psi}}}\right)\sqrt{\frac{\mel{\psi}{L_l^\dagger L_l}{\psi}}
    {\mel{\psi}{L_l^\dagger M_m^\dagger M_m L_l}{\psi}}} \\
  &= \frac{M_mL_l\ket{\psi}}{\sqrt{\mel{\psi}{L_l^\dagger M_m^\dagger M_m L_l}
    {\psi}}} \\
  &= \frac{N_{lm}\ket{\psi}}{\sqrt{\mel{\psi}{N_{lm}^\dagger N_{lm}}{\psi}}}.
\end{align*}

\section{Distinguishing quantum states}

Consider a game where Alice choose a state $\ket{\psi_i} (1 \leq i \leq n)$
from some fixed set of states known to both Alice and Bob. She gives the state
$\ket{\psi_i}$ to Bob, whose task it is to identify the index $i$ of the state
Alice has given him.

Suppose the states $\ket{\psi_i}$ are orthonormal. Then Bob can do a quantum
measurement to distinguish these states, using the following procedure. Define
measurement operators $M_i \equiv \ketbra{\psi_i}{\psi_i}$, one for each
possible index $i$, and an additional measurement operator $M_0$ defined as the
positive square root of the positive operator $I - \sum_{i \neq 0}
\ketbra{\psi_i}{\psi_i}$. These operators satisfy the completeness relation,
and if the state $\ket{\psi_i}$ is prepared then $p(i) = \mel{\psi_i}{M_i}
{\psi_i} = 1$, so the result $i$ occurs with certainty. Thus, it is possible
to reliably distinguish the orthonormal states $\ket{\psi_i}$.

By contrast, if the states $\ket{\psi_i}$ are not orthonormal then we can prove
that there is \emph{no quantum measurement capable of distinguishing the
states}.

\begin{theorem}
  Non-orthogonal states cannot be reliably distinguished through quantum
  measurements.
\end{theorem}

\begin{proof}
  A proof by contradiction shows that no measurement distinguishing the
  non-orthogonal states $\ket{\psi_1}$ and $\ket{\psi_2}$ is possible. Suppose
  such a measurement is possible. If the state $\ket{\psi_1}$ ($\ket{\psi_2}$)
  is prepared then the probability of measuring $j$ such that $f(j) = 1$ ($f(j)
  = 2$) must be 1. Defining $E_i \equiv \sum_{j: f(j) = i}M_j^\dagger M_j$,
  these observations may be written as: \begin{equation*}
    \mel{\psi_1}{E_1}{\psi_1} = 1; \mel{\psi_2}{E_2}{\psi_2} = 1.
  \end{equation*}
  Since $\sum_iE_i = I$ it follows that $\sum_i\mel{\psi_1}{E_i}{\psi_1} = 1$,
  and since $\mel{\psi_1}{E_1}{\psi_1} = 1$, we must have $\mel{\psi_1}{E_2}
  {\psi_1} = 0$, and thus $\sqrt{E_2}\ket{\psi_1} = 0$. Suppose we decompose
  $\ket{\psi_2} = \alpha\ket{\psi_1} + \beta\ket{\phi}$, where $\ket{\phi}$ is
  orthonormal to $\ket{\psi_1}$, $|\alpha|^2 + |\beta|^2 = 1$, and $|\beta| <
  1$ since $\ket{\psi_1}$ and $\ket{\psi_2}$ are not orthogonal. Then
  $\sqrt{E_2}\ket{\psi_2} = \beta\sqrt{E_2}\ket{\phi}$, which implies a
  contradiction, as \begin{equation*}
    \mel{\psi_2}{E_2}{\psi_2} = |\beta|^2\mel{\phi}{E_2}{\phi} \leq |\beta|^2
      < 1,
  \end{equation*} where the second last inequality follows from the observation
  that \begin{equation*}
    \mel{\phi}{E_2}{\phi} \leq \sum_i\mel{\phi}{E_i}{\phi} = \braket{\phi}
      {\phi} = 1.
  \end{equation*}
\end{proof}

\section{Projective measurements}

\paragraph{Projective measurements.} A projective measurement is described by
an \emph{observable}, $M$, a Hermitian operator on the state space of the
system being observed. The observable has a spectral decomposition,
\begin{equation*}
  M = \sum_m mP_m,
\end{equation*} where $P_m$ is the projector onto the eigenspace of $M$ with
eigenvalue $m$. The possible outcomes of the measurement correspond to the
eigenvalues, $m$, of the observable. Upon measuring the state $\ket{\psi}$, the
probabiltiy of getting result $m$ is given by \begin{equation*}
  p(m) = \mel{\psi}{P_m}{\psi}.
\end{equation*} Given that outcome $m$ occurred, the state of the quantum
systenm immediately after the measurement is \begin{equation*}
  \frac{P_m\ket{\psi}}{\sqrt{p(m)}}.
\end{equation*}

Projective measuremnts can be understood as a special case of Postulate 3.
Suppose the measurement operators in Postulate 3, in addition to satisftying
the completeness relation $\sum_m M_m^\dagger M_m = I$, also satisfy the
conditions that $M_m$ are orthogonal projectors, that is, the $M_m$ are
Hermitian, and $M_mM_{m'} = \delta_{m, m'}M_m$. With these additional
restrictions, Postulate 3 reduces to a projective measurement as just defined.

The average value of the measurement is \begin{align*}
  \langle M \rangle &= \sum_m mp(m) \\
    &= \sum_m m\mel{\psi}{P_m}{\psi} \\
    &= \bra{\psi}\left(\sum_m mP_m\right)\ket{\psi} \\
    &= \mel{\psi}{M}{\psi};
\end{align*} the average value of the observable $M$ is often written $\langle
M \rangle \equiv \mel{\psi}{M}{\psi}$. From this formula for the average
follows a formula for the standard deviation associated to observations of $M$,
\begin{equation*}
  [\Delta(M)]^2 = \langle (M - \langle M \rangle)^2 \rangle = \langle M^2
    \rangle - \langle M \rangle^2.
\end{equation*}

\paragraph{\cite{mikeandike} Exercise 2.58:} Suppose we prepare a quantum
system in an eigenstate $\ket{\psi}$ of some observable $M$, with corresponding
eigenvalue $m$. What is the average observed value of $M$, and the standard
deviation?

\paragraph{Solution:} The average observed value of $M$ is \begin{equation*}
  \langle M \rangle = \mel{\psi}{M}{\psi} = \mel{\psi}{m}{\psi} =
    m\braket{\psi}{\psi} = m.
\end{equation*} Similarly, the average observed value of $M^2$ is
\begin{equation*}
  \langle M^2 \rangle = \mel{\psi}{M^2}{\psi} = \mel{\psi}{m^2}{\psi} =
    m^2\braket{\psi}{\psi} = m^2.
\end{equation*} It is clear that the standard deviation is $\Delta(M) = \sqrt{
  \langle M^2 \rangle - \langle M \rangle^2} = \sqrt{m^2 - m^2} = 0$.

\paragraph{} This formulation of measurement and standard deviations in terms
of observables gives rise in an elegant way to results such as the
\emph{Heisenberg uncertainty principle}.

\begin{proof}[Heisenberg uncertainty principle]
  Suppose $A$ and $B$ are two Hermitian operators, and $\ket{\psi}$ is a
  quantum state. Suppose $\mel{\psi}{AB}{\psi} = x + iy$, where $x$ and $y$ are
  real. Note that $\mel{\psi}{[A, B]}{\psi} = 2iy$ and $\mel{\psi}{\{A, B\}}
  {\psi} = 2x$. This implies that \begin{equation*}
    |\mel{\psi}{[A, B]}{\psi}|^2 + |\mel{\psi}{\{A, B\}}{\psi}|^2 =
      4|\mel{\psi}{AB}{\psi}|^2.
  \end{equation*} By the Cauchy-Schwarz inequality \begin{equation*}
    |\mel{\psi}{AB}{\psi}|^2 \leq \mel{\psi}{A^2}{\psi}\mel{\psi}{B^2}{\psi},
  \end{equation*} which combined with the above equation and dropping a
  non-negative term gives \begin{equation*}
    |\mel{\psi}{[A, B]}{\psi}|^2 \leq 4\mel{\psi}{A^2}{\psi}\mel{\psi}{B^2}
      {\psi}.
  \end{equation*} Suppose $C$ and $D$ are two observables. Substituting $A = C
  - \langle C \rangle$ and $B = D - \langle D \rangle$ into the last equation,
  we obtain Heisenberg's uncertainty principle as it is usually stated:
  \begin{equation*}
    \Delta(C)\Delta(D) \geq \frac{|\mel{\psi}{[C, D]}{\psi}|}{2}.
      \label{eq:hei-unc-prin}
  \end{equation*}

  The correct interpretation of the uncertainty principle is that if we prepare
  a large number of quantum systems in identical states, $\ket{\psi}$, and then
  perform measurements of $C$ on some of those systems, and of $D$ in others,
  then the standard deviation $\Delta(C)$ of the $C$ results times the standard
  deviation $\Delta(D)$ of the $D$ results for $D$ will satisfy the inequality
  \eqref{eq:hei-unc-prin}.
\end{proof}

Rather than giving an observable to describe a projective measurement, often
people simply list a complete set of orthogonal projectors $P_m$ satisfying the
relations $\sum_m P_m = I$ and $P_mP_{m'} = \delta_{mm'}P_m$. The corresponding
observable implicit in this usage is $M = \sum_m mP_m$. Another widely used
phrase, to `measure in a basis $\ket{m}$', where $\ket{m}$ forms an orthonormal
basis, simply means th perform the projective measurement with projectors $P_m
= \ketbra{m}{m}$.

\paragraph{} Suppose $\vec{v}$ is any real three-dimensional unit vector. Then
we can define an observable: \begin{equation*}
  \vec{v} \cdot \vec{\sigma} \equiv v_1\sigma_1 + v_2\sigma_2 + v_3\sigma_3.
\end{equation*} Measurement of this observable is sometimes referred to as a
`measurement of spin along the $\vec{v}$ axis', for historical reasons.

\paragraph{\cite{mikeandike} Exercise 2.59:} Suppose we have qubit in the state
$\ket{0}$, and we measure the observable $X$. What is the average value of $X$?
What is the standard deviation of $X$?

\paragraph{Solution:} The average observed value of $X$ is \begin{equation*}
  \langle X \rangle = \mel{0}{X}{0} = \braket{0}{1} = 0,
\end{equation*} while the average observed of $X^2$ is \begin{equation*}
  \langle X^2 \rangle = \mel{0}{X^2}{0} = \braket{1}{1} = 1.
\end{equation*} Hence, the standard deviation of $X$ is $\Delta(X) = \sqrt{
\langle X^2 \rangle - \langle X \rangle^2} = \sqrt{1 - 0^2} = 1$.

\subsection{POVM Measurements}

\paragraph{Positive Operator-Valued Measure (POVM).} Suppose a measurement
described by measurement operators $M_m$ is performed upon a quantum system in
the state $\ket{\psi}$. Then the probability of outcome $m$ is given by $p(m) =
\mel{\psi}{M_m^{\dagger}M_m}{\psi}$. Suppose we define \begin{equation*}
  E_m \equiv M_m^{\dagger}M_m.
\end{equation*} Then from Postulate 3 and elementary linear algebra, $E_m$ is a
positive operator such that $\sum_mE_m = I$ and $p(m) = \mel{\psi}{E_m}{\psi}$.
Thus the set of operators $E_m$ are sufficient to determine the probabilities
of the different measurement outcomes. The operators $E_m$ are known as the
\emph{POVM elements} associated with the measurement. The complete set
$\{E_m\}$ is known as a \emph{POVM}.

\paragraph{\cite{mikeandike} Exercise 2.62:} Show that any measurement where
the measurement operators and the POVM elements coincide is a projective
measurement.

\paragraph{Solution:} A measurement operator $M_m$ would be defined as a
POVM element $E_m = M_m^{\dagger}M_m$, which leads to \begin{equation*}
  \mel{\psi}{E_m}{\psi} = \mel{\psi}{M_m}{\psi} \geq 0
\end{equation*} for all $\ket{\psi}$. As $M_m$ is a positive operator, it must
be Hermitian. Therefore, \begin{equation*}
  E_m = M_m^{\dagger}M_m = M_mM_m = M_m^2 = M_m
\end{equation*} which, by definition, shows that $M_m$ be a projective
measurement.

\paragraph{\cite{mikeandike} Exercise 2.63:} Suppose a measurement is described
by measurement operators $M_m$. Show that there exist unitary operators $U_m$
such that $M_m = U_m\sqrt{E_m}$, where $E_m$ is the POVM associated to the
measurement.

\paragraph{Solution:} Noting that $U_m^{\dagger}U_m = I$, \begin{align*}
  M_m^{\dagger}M_m &= \sqrt{E_m}U_m^{\dagger}U_m\sqrt{E_m} \\
    &= \sqrt{E_m}I\sqrt{E_m} \\
    &= E_m.
\end{align*}

\subsection{Phase}

\paragraph{Phase.} Two amplitudes, $a$ and $b$, \emph{differ by a relative
phase} if there is a real $\theta$ such that $a = \exp(i\theta)b$.

[more elaboration needed, the elaboration in my book talks about global phase
vs relative phase but idk if we'll want to put that in]

\paragraph{\cite{mikeandike} Exercise 2.65:} Express the states $(\ket{0} +
\ket{1})/\sqrt{2}$ and $(\ket{0} - \ket{1})/\sqrt{2}$ in a basis in which they
are not the same up to a relative phase shift.

[idk what the soln for this is supposed to mean either]

